\documentclass[12pt, leqno, a4paper]{article}

\makeatletter
\newcommand{\ps@pagenumberfoot}{
\renewcommand{\@oddfoot}{\hfil\thepage}
\renewcommand{\@evenfoot}{\thepage\hfil}
}
\makeatother

\input{Tyyli.sty}
\pagestyle{plain}

\begin{document}

\title{Määrittelydokumentti}
\author{Tomi Heiskanen}

\thispagestyle{empty}
\maketitle

\section*{Tietorakennevertailut}

Työssä on tarkoitus vertailla neljää eri tietorakennetta: AVL-puuta, punamusta
puuta, B-puuta ja binäärihakupuuta. Tarkoitus olisi tutkia ja vertailla, missä 
tilanteessa kukin tietorakenne olisi sopivin. Kullakin hakupuulla on erilaiset toteutus
tavat ja siten voinee olettaa, että toinen puu voi toimia toista paremmin
samassa tilanteessa. Tietorakenteet ovat valittu satunnaisesti. Olen kuullut
joskus kyseisistä rakenteista joissain yhteyksissä ja nyt googlaamalla selvittänyt
minkälaisia ne suurinpiirtein ovat.

Kaikilla tietorakenteille on ilmoitettu $O$-analyysin tulokset. AVL-puulla haulle,
lisäykselle ja poistolle se on $O(\log n)$ keskimääräisessä ja pahimmassa
tapauksessa. Punamusta puulle haku, lisäys ja poisto voidaan tehdä samassa
ajassa $O(\log n)$, kuten B-puulle. Binäärihakupuulla pahin tapaus voi olla
$O(n)$. Vaikka tulokset ovat samat kolmelle ensimmäiselle, ja hyvässä tapauksessa
neljännelle, niin hakupuilla on erilaiset toteutustavat. Tilanteesta riippuen 
kannattanee miettiä mitä rakennetta käyttää, toinen voi toimia toista nopeammin 
vaikka $O$-analyysin tulokset ovatkin samat.

Tietorakenteita tultanee testaamaan generoimalla satunnaisia kokonaislukuja ja
lisäämällä niitä tietorakenteisiin, sekä hakemalla ja poistamalla satunnaisia
lukuja, jos luku löytyy tietorakenteesta. Lisäyksien, hakujen ja poistojen vertailua
suoritetaan myös toteutusta koodatessa ja toteutuksesta voi myös arvioida,
minkälaisiin tilanteisiin tietorakenne sopisi.

\nocite{AVL}
\nocite{PM}
\nocite{Bin}
\nocite{B-puu}
\renewcommand{\refname}{Lähteet}
\bibliography{tira}
\bibliographystyle{plain}
\end{document}

