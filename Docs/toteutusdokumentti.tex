\documentclass[12pt,a4paper,leqno,titlepage,twoside]{article}

\makeatletter
\newcommand{\ps@pagenumberfoot}{
\renewcommand{\@oddfoot}{\hfil\thepage}
\renewcommand{\@evenfoot}{\thepage\hfil}
}
\makeatother

\input{Tyyli.sty}
\pagestyle{pagenumberfoot}

\begin{document}

\title{Toteutusdokumentti}
\author{Tomi Heiskanen}
\maketitle

\newpage
\section*{Tietorakenteet}

Tietorakenteet on toteutettu omina java-luokkina. Kaikissa puissa on solmuksi
nimetty alkio ja se on toteutettu omana luokkanaan ja siinä on tarvittavat
ominaisuudet jokaiselle puulle.

\section*{O-analyysit}

\subsection*{Binäärihakupuu}

\begin{description}
\item{\textbf{Lisääminen:}} Koska metodissa olevan loopin operaatiot ovat vakioaikaisia.
ja looppi suoritetaan korkeintaan $n$-kertaa, sillä lisäykset voivat pahimmassa
tapauksessa tapahtua suuruusjärjestyksessä, saadaan lisäämisen aikavaativuudeksi
$O(n)$.

\item{\textbf{Poistaminen:}} Metodissa olevat operaatiot ovat vakioaikaisia. Solmun
etsiminen suoritetaan toisella metodilla ja se on rekursiivinen. Rekursio
suoritetaan pahimmillaan $n$- kertaa, sillä lisäämiset voivat tapahtua suuruusjärjestyksessä
tai jos puuhun lisätään ja poistetaan sopivasti siten, että puu vastaa listaa.
Poistamiselle saadaan siten aikavaativuudeksi $O(n)$.

\item{\textbf{Etsiminen:}} Etsiminen suoritetaan rekursiivisella metodilla ja
rekursio suoritetaan pahimmassa tapauksessa $n$-kertaa, kuten edellä todettiin. 
Etsimisen aikavaativuudeksi saadaan $O(n)$.

\item{\textbf{Pienimmän ja suurimman alkion etsiminen:}} Pienin ja suurin alkio
löytyvät aina lehdestä tai juuresta ja siten metodissa oleva looppi suoritetaan 
pahimmassa tapauksessa $n$-kertaa. Aikavaativuudeksi saadaan tällöin $O(n)$.
\end{description}

\subsection*{AVL-puu}

\begin{description}
\item{\textbf{Lisääminen:}} Lisääminen tapahtuu ensin kuten binäärihakupuulle
ja sen aikavaatimus on $O(h)$, missä $h$ on korkeus ja koska AVL-puu on tasapainoinen,
niin lisäämisen aikavaatimus on $O(\log n)$. Koska lisäämisen jälkeen suoritettavat 
kierrot ovat vakioaikaisia ja niitä suoritetaan korkeintaan kaksi kertaa ja pahimmassa 
tapauksessa kuljetaan lisätystä solmusta juureen, niin tämän aikavaativuudeksi 
saadaan $O(\log n)$. Lisääminen koostuu siis kahdesta aikavaativuudeltaan $O(\log n)$
osasta ja siten lisäämisen aikavaativuus on $O(\log n)$.

\item{\textbf{Poistaminen:}} Poistaminen tapahtuu ensin kuten binäärihakupuulle
ja sen aikavaatimus on $O(h)$, missä $h$ on korkeus ja koska AVL-puu on tasapainoinen,
niin poistamisen aikavaatimus on $O(\log n)$. Koska poistamisen jälkeen suoritettavat
kierrot ovat vakioaikaisia ja niitä suoritetaan korkeintaan kaksi kertaa ja pahimmassa
tapauksessa kuljetaan poistetusta solmusta juureen, niin tämän aikavaativuudeksi
saadaan $O(\log n)$. Poistaminen koostuu siis kahdesta aikavaativuudeltaan $O(\log n)$
osasta ja siten poistamisen aikavaativuus on $O(\log n)$.

\item{\textbf{Etsiminen:}} Etsiminen suoritetaan rekursiivisella metodilla ja
rekursio suoritetaan korkeintaan $\log n$ kertaa. Etsimisen aikavaativuudeksi
saadaan $O(\log n)$.

\item{\textbf{Pienimmän ja suurimman alkion etsiminen:}} Pienin ja suurin alkio
löytyy aina lehdestä ja metodissa oleva looppi suoritetaan siis $\log n$-kertaa.
Aikavaativuudeski saadaan tällöin $O(\log n)$.
\end{description}

\subsection*{AVL-puu}

\begin{description}
\item{\textbf{Lisääminen:}} Lisääminen tapahtuu ensin kuten binäärihakupuulle
ja sen aikavaatimus on $O(h)$, missä $h$ on korkeus ja koska AVL-puu on tasapainoinen,
niin lisäämisen aikavaatimus on $O(\log n)$. Koska lisäämisen jälkeen suoritettavat
kierrot ovat vakioaikaisia ja niitä suoritetaan korkeintaan kaksi kertaa ja pahimmassa
tapauksessa kuljetaan lisätystä solmusta juureen, niin tämän aikavaativuudeksi
saadaan $O(\log n)$. Lisääminen koostuu siis kahdesta aikavaativuudeltaan $O(\log n)$
osasta ja siten lisäämisen aikavaativuus on $O(\log n)$.

\item{\textbf{Poistaminen:}} Poistaminen tapahtuu ensin kuten binäärihakupuulle
ja sen aikavaatimus on $O(h)$, missä $h$ on korkeus ja koska AVL-puu on tasapainoinen,
niin poistamisen aikavaatimus on $O(\log n)$. Koska poistamisen jälkeen suoritettavat
kierrot ovat vakioaikaisia ja niitä suoritetaan korkeintaan kaksi kertaa ja pahimmassa
tapauksessa kuljetaan poistetusta solmusta juureen, niin tämän aikavaativuudeksi
saadaan $O(\log n)$. Poistaminen koostuu siis kahdesta aikavaativuudeltaan $O(\log n)$
osasta ja siten poistamisen aikavaativuus on $O(\log n)$.

\item{\textbf{Etsiminen:}} Etsiminen suoritetaan rekursiivisella metodilla ja
rekursio suoritetaan korkeintaan $\log n$ kertaa. Etsimisen aikavaativuudeksi
saadaan $O(\log n)$.

\item{\textbf{Pienimmän ja suurimman alkion etsiminen:}} Pienin ja suurin alkio
löytyy aina lehdestä ja metodissa oleva looppi suoritetaan siis $\log n$-kertaa.
Aikavaativuudeski saadaan tällöin $O(\log n)$.
\end{description}

\subsection*{Punamustapuu}

\begin{description}
\item{\textbf{Lisääminen:}} Lisääminen tapahtuu ensin kuten binäärihakupuulle
ja sen aikavaatimus on $O(h)$, missä $h$ on korkeus ja koska punamustapuu on tasapainoinen,
niin lisäämisen aikavaatimus on $O(\log n)$. Koska lisäämisen jälkeen suoritettavat
kierrot ovat vakioaikaisia ja niitä suoritetaan korkeintaan kaksi kertaa ja pahimmassa
tapauksessa kuljetaan lisätystä solmusta juureen, niin tämän aikavaativuudeksi
saadaan $O(\log n)$. Lisääminen koostuu siis kahdesta aikavaativuudeltaan $O(\log n)$
osasta ja siten lisäämisen aikavaativuus on $O(\log n)$.

\item{\textbf{Poistaminen:}} Poistaminen tapahtuu ensin kuten binäärihakupuulle
ja sen aikavaatimus on $O(h)$, missä $h$ on korkeus ja koska AVL-puu on tasapainoinen,
niin poistamisen aikavaatimus on $O(\log n)$. Koska poistamisen jälkeen suoritettavat
kierrot ovat vakioaikaisia ja niitä suoritetaan korkeintaan kaksi kertaa ja pahimmassa
tapauksessa kuljetaan poistetusta solmusta juureen, niin tämän aikavaativuudeksi
saadaan $O(\log n)$. Poistaminen koostuu siis kahdesta aikavaativuudeltaan $O(\log n)$
osasta ja siten poistamisen aikavaativuus on $O(\log n)$.

\item{\textbf{Etsiminen:}} Etsiminen suoritetaan rekursiivisella metodilla ja
rekursio suoritetaan korkeintaan $\log n$ kertaa. Etsimisen aikavaativuudeksi
saadaan $O(\log n)$.

\item{\textbf{Pienimmän ja suurimman alkion etsiminen:}} Pienin ja suurin alkio
löytyy aina lehdestä ja metodissa oleva looppi suoritetaan siis $\log n$-kertaa.
Aikavaativuudeski saadaan tällöin $O(\log n)$.
\end{description}

\section*{Suorituskyky}

\begin{table}
\begin{tabular}{|p{0.2\textwidth}|p{0.2\textwidth}|p{0.2\textwidth}|p{0.2\textwidth}|p{0.2\textwidth}|}
  \hline
  Toiminto & Bi\-nää\-ri\-ha\-ku\-puu & AVL-puu & Pu\-na\-mus\-ta\-puu & B-puu \\
  \hline
  \hline
  Lisääminen & ms & ms & ms & ms \\
  Poistaminen & ms & ms & ms & ms \\
  Etsiminen & ms & ms & ms & ms \\
  Pienimmän ja suurimman etsiminen & ms & ms & ms & ms \\
  \hline
\end{tabular}
\end{table}


\end{document}
