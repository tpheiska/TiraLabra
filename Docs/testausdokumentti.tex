\documentclass[12pt,a4paper,leqno,titlepage]{article}

\input{Tyyli.sty}

\begin{document}
\title{Testausdokumentti}
\author{Tomi Heiskanen}
\maketitle

\section*{Testit}

Tietorakenteita on testattu kahdella tavalla, ensiksi koodia kirjoittaessa on
suoritettu JUnit-testejä ja saatu toimiva koodi ja sen jälkee on suoritttu
suorituskykytestaukset. JUnit-testeissä on lisätty, poistettu ja etsitty solmuja
ja tulostettu puun solmujen tietosisältöjä järjestyksessä. Suorituskykytestauksessa
on generoitu suuri määrä kokonaislukuja taulukkoon ja sitten lisätty puihin järjestyksessä.
Lisäämiseen kulunut aika on mitattu. Lisäksi on mitattu aika, joka kuluu
tietyn arvon etsimiseen, poistamiseen ja suurimman ja pienimmän alkion etsimiseen.

Suorituskykytestaukset tehdään suorittamalla testipääohjelma kustakin puusta
erikseen. Suorittamalla se, luodaan puu ja siihen aletaan lisäämään solmuja. Ensin
luodaan kokonaislukutaulukko, joka on järjestyksessä, ja aletaan lisäämään solmuja,
joiden tietokentän arvoksi tulee koknaisluku, puuhun. Etsimisen testaamiseksi
luodaan kokonaislukutaulukko, jossa on satunnaisesti tuotettuja kokonaislukuja
10000 ja katsotaan kuinka kauan niiden etsimiseen menee aikaa. Suurimman ja
pienimmän luvun etsiminen tapahtui sen verran nopeasti, että jokaisella alkioiden
määrällä sain ajaksi 0 ms.

\begin{figure}[!h]
  \centering
  % Requires \usepackage{graphicx}
  \includegraphics[width=0.8\textwidth]{binaarihakupuu.eps}\\
  \caption{Binäärihakupuuhun lisäämiseen ja siitä hakemiseen kuluvat ajat}
\end{figure}

Binäärihakupuuhun lisäämisen, poistamisen ja etsimisen aikavaativuus on pahimmillaan
$O(n)$. Kuvan perusteella näin näyttäisi olevan. Kuitenkin kun lisättävien alkioiden
määrä lisääntyy huomattavasti, aikavaativuus näyttäisi olevan pahempi, kuin $O(n)$. 
En osaa selittää tätä, mutta arvelisin sen liittyvän muistin määrään. Etsimisen
aikavaativuudelle näin ei näyttäisi tapahtuvan.

\begin{figure}[!h]
  \centering
  % Requires \usepackage{graphicx}
  \includegraphics[width=0.8\textwidth]{avl.eps}\\
  \caption{AVL-puuhun lisäämiseen ja siitä hakemiseen kuluvat ajat}
\end{figure}

AVL-puun lisäämisen, poistamisen ja etsimisen aikavaativuus on pahimmillaan $O(\log n)$
ja näin näyttäisi testien alkupään mukaan olevan. AVL-puun testissä käy samalla
tavalla kuin binäärihakupuulla, jolla solmujen määrän lisääntyessä huomattavasti
solmujen lisäämisen aikavaativuus näyttäisi olevan pahempi kuin se on. Arvelisin 
syyn olevan saman kuin binäärihakupuulla.

\begin{figure}[!h]
  \centering
  % Requires \usepackage{graphicx}
  \includegraphics[width=0.8\textwidth]{punamusta.eps}\\
  \caption{Punamustapuuhun lisäämiseen ja siitä hakemiseen kuluvat ajat}
\end{figure}

Punamustapuun testit menevät edellisten mukaan. Suurilla solmujen määrällä
testitulokset lisättävien solmujen osalta poikkeaa aikavaativuusanalyysin tuloksesta.
Syy lienee sama kuin edellisilläkin puilla.

\end{document}
