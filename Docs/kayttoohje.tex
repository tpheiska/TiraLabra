\documentclass[12pt,a4paper,leqno,titlepage]{article}

\input{Tyyli.sty}

\begin{document}
\title{Käyttöohje}
\author{Tomi Heiskanen}
\maketitle

\section*{Puiden käyttö}

Tietorakenteita voidaan käyttää luomalla oliot niistä. Solmun tietokentän arvo
on kokonaisluku, mutta sen voisi muutta miksi tahansa vertailtavaksi olioksi.
Tällöin pitää huolehtia että puun hakemisen, lisäämisen ja poistamisen yhteydessä
vertaillaan sitten avainarvoja, eikä kokonaislukuja suoraan kuten nyt. Jokaisella
puulla on testipääohjelmansa ja sen voi suorittaa ajamalla käännetyn java-koodin.

Testipääohjelmassa lisättävien alkioiden määrää voidaan muuttaa muuttamalla
\verb+maara+ -muuttujan arvoa. Se pitää kirjoittaa itse testipääohjelmaan ja muutoksen
jälkeen puu pitää kääntää uudelleen. JUnit testit ovat \\
TiraLabra\_maven/Tietorakennevertailut/src/test/java-kansiossa. Suorituskykytestit 
suoritetaan ajamalla testipääohjelma
\end{document}
